% A section with detailing each member’s taks in the project + l'organisation interne
\chapter{Méthodologie}
    Pour réaliser ce projet, nous avons commencé par réaliser quelques recherches pour trouver des idées de comment nous pouvons atteindre nos objectifs.
    \\
    Après avoir récolté quelques idées, nous avons décidé de nous répartir les tâches en fonction des affinités de chacun avec les différentes parties du projet.
    \\
    Nous avons alors commencé à réaliser les différentes parties du projet en parallèle, en discutant régulièrement pour faire le point sur l'avancement du projet et afin de s'entraider si besoin mais surtout de comprendre les différentes parties du projet.
    \\
    Nous avons également utilisé un dépôt git pour pouvoir travailler sur les mêmes fichiers en même temps, et pour pouvoir revenir à une version antérieure si besoin ce qui nous aura été utile à plusieurs reprises.
    \\
    À plusieurs reprises nous avons décidé de pivoter, pour cela nous avons dû parfois grandement modifié notre code et robot, nous pouvions donc créer une nouvelle branche afin de réaliser nos expérimentations sans risquer de casser le code fonctionnel.
    \\
    Pour travailler, la plupart du temps nous nous sommes réunis à l'INSA pendant les heures de TP, mais aussi souvent en dehors des heures de cours pour trvouer le temps de corriger les plus gros problèmes.
    \\

