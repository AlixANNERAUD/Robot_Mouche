% A section with detailing each member’s taks in the project + l'organisation interne
\chapter{Méthodologie}
    Pour réaliser ce projet, nous avons commencé par réaliser quelques recherches afin de trouver des idées pour atteindre nos objectifs et respecter le cahier des charges.
    \\
    Après avoir récolté quelques idées, nous nous sommes réparti les tâches en fonction des affinités de chacun avec les différentes parties du projet.
    \\
    Nous avons alors commencé à travailler en parallèle sur les différentes parties du projet, tout en discutant régulièrement pour faire le point sur l'avancement et s'entraider si besoin. Cela nous a surtout permis de comprendre tout le travail réaliser par l'ensemble du groupe.
    \\
    Nous avons également utilisé un dépôt git pour pouvoir travailler sur les mêmes fichiers en même temps, ainsi que pour être capable de revenir à une version antérieure (ce qui nous aura été utile à plusieurs reprises).
    Une CI (intégration continue) a également été mise en place afin de vérifier que le code compile et que les tests passent à chaque push sur le git afin d'éviter de casser le code sur le dépôt.
    \\
    À plusieurs reprises nous avons "pivoter" et changer d'axes de travail (que ce soit à propos du code ou de nos composants). C'est pour cela que nous avons parfois grandement modifié l'ensemble de notre projet. Nous pouvions alors créer une nouvelle branche afin de réaliser nos expérimentations sans risquer de perdre un code fonctionnel.
    \\
    Pour travailler, la plupart du temps nous nous sommes réunis à l'INSA pendant les heures de TP, mais également souvent en dehors des heures de cours pour trouver le temps de corriger les plus gros problèmes.
    \\

