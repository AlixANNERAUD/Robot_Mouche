
\chapter{Conclusion}
% Partie dans laquelle on revient sur ces difficultés, on essaie de les raisonner
% (leur trouver une raison) et faire comme si le projet avait été une source incroyable d'apprentissage pour nous

Pour conclure ce rapport nous pouvons revenir sur tous ce que projet nous a apporté.

Sur le plan purement technique nous avons beaucoup appris car ce robot est notre première réalisation complexe en électronique (pour la majorité du groupe). Nous avons pu mettre en pratique nos connaissances de cours, à l'instar de l'utilisation de Wiring Pi, du PWM, du protocole I2C et des différents composants utilisés durant les sessions en labo, mais nous sommes également allés plus loin en découvrant de nouvelles choses. Nous pouvons notamment citer la gestion des threads, l'utilisation de nouveaux composants comme le DAC/Amplifier ou encore le LIDAR.

Même nos échecs comme l'utilisation des capteurs QTR nous a permis dans en apprendre plus sur de nouveaux sujets. De plus, certains membres du groupe ont pu découvrir de nouveaux outils pour la gestion de projet tel que PlatformIO ou easyEDA.

Le projet a aussi été une source d'apprentissage sur le plan organisationnel. La longueur de celui-ci et le nombre important de fonctionnalités que nous voulions réaliser nous a obligé à fixer une organisation bien définie dès le début du projet. Cependant, il est important de remarquer que se sont les difficultés rencontrées au fur et à mesure des séances qui constituent les meilleures expériences pour nos futurs projets. En effet, notre envie d'améliorer le plus possible notre robot nous a poussé à utiliser dès le départ des composants totalement nouveaux pour nous et de nous écarter des lignes toutes tracées du sujet. En raison de cette dynamique très innovante nous avons pris beaucoup de retards suite aux difficultés rencontrées, notamment sur les capteurs QTR dont l'abandon nous a forcé à changer successivement notre capteur de ligne, l'emplacement de nos roues et enfin notre algorithme suiveur de ligne et tout ceci durant les dernières séances.

Pour conclure, notre robot n'est pas parfait (notamment sur les passages difficiles du parcours de lignes) mais le projet nous a apporté de vraies connaissances ainsi qu'une réelle expérience de gestion de projet qui nous permettra d'être bien meilleurs sur nos prochaines réalisations. Enfin, nous avons tout de même pris beaucoup de plaisir à travailler ensemble, même si le projet nous a demandé beaucoup de temps et d'efforts.