% An introduction with the project objectives
\chapter{Introduction}

\section{Contexte}

Dans le cadre du cours d'électronique pour système embarqué,
nous avons eu pour projet de réaliser un robot télécommandé et suiveur de ligne (voir \ref{cahier_des_charges}).
Ce rapport présente notre robot, les différentes étapes de sa conception ainsi que les difficultés que nous avons rencontrées.

\section{Cahier des charges} \label{cahier_des_charges}

Le robot possède deux modes de fonctionnement :

\subsection{Mode télécommandé}

Le robot est contrôlé par un utilisateur à l'aide d'une télécommande.
Il peut se déplacer dans les quatre directions (avant, arrière, gauche, droite)

\subsection{Mode suiveur de ligne}

Le robot est capable de suivre une ligne noire au sol sur un fond blanc.
Cette ligne peut être droite, courbé ou être interrompue par des intersections.
Le robot dans ce mode doit également détecter les obstacles et s'arrêter et émettre un signal sonore ou lumineux.

\section{Outils}

Afin de concevoir et réaliser notre robot, nous avons utilisé les outils suivants :
\begin{itemize}
    \item Le language C++ (version de 2017) pour le développement du programme du robot avec les bibliothèques suivantes :
    \begin{itemize}
        \item \hyperlink{https://github.com/WiringPi/WiringPi}{WiringPi} pour le contrôle des GPIO de notre Raspberry Pi.
        \item \hyperlink{https://github.com/yhirose/cpp-httplib}{cpp-httplib} pour la partie serveur web de notre robot.
    \end{itemize}
    \item Python 3 pour la récupération des données de la caméra.
    \item HTML, CSS et JavaScript pour le développement de l'interface web de contrôle du robot.
    \item Latex pour la rédaction de ce rapport.
    \item Visual Studio Code couplé avec les extensions suivantes :
    \begin{itemize}
        \item PlatformIO : un IDE pour le développement embarqué.
        \item C/C++ : un plugin pour le language C++.
        \item Latex Workshop : un plugin pour le language Latex.
    \end{itemize}
    \item Git et GitHub pour la gestion de version, l'hébergement du code source et l'intégration continue.
\end{itemize}