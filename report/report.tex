\documentclass[a4paper, 12pt]{report}

% Packages
\usepackage[utf8]{inputenc}
\usepackage[T1]{fontenc}
\usepackage[french]{babel}
\usepackage{graphicx}
\usepackage{float}
\usepackage{hyperref}
\usepackage{fancyhdr}
\usepackage{titlesec}
\usepackage{lipsum}

% Mise en page
\pagestyle{fancy}
\fancyhf{}
\setlength{\headheight}{15pt}
\renewcommand{\headrulewidth}{0.5pt}
\renewcommand{\footrulewidth}{0pt}
\lhead{\leftmark}
\rhead{\thepage}
\renewcommand{\chaptermark}[1]{\markboth{\MakeUppercase{#1}}{}}
\titleformat{\chapter}[display]{\normalfont\huge\bfseries}{\chaptertitlename\ \thechapter}{20pt}{\Huge}

% Informations du document
\title{Test Rapport de projet d'électronique pour système embarqué}
\author{Simon GIRARD - Dimitri TIMOZ - Mathis SAUNIER - Alix ANNERAUD}
\date{\today}

\begin{document}

% Page de garde
\begin{titlepage}
\maketitle{}
\end{titlepage}

% Table des matières
\tableofcontents

% Chapitres
% An introduction with the project objectives
\chapter{Introduction}
\lipsum[1-2]
% 
\chapter{Contexte}
\lipsum[3-4]

\chapter{Conception}

% A complete fritzing circuit with all the wiring of the different sensors, actuators and components for display
% + A detailed explanation of your wiring choices as well as any additional technical choices (power, use of resistors, diodes, etc.)
% + A detailed explanation of the chosen components role and operating principle
\section{Electronique}
\subsection{Choix des composants}
\subsection{Cablâge des composants}
\subsection{Schéma EasyEDA}
% A detailed section of your code: approach, structure, etc.
\section{Code}
\subsection{Architecture du programme}
\subsection{Développement des fonctions métiers du robot}
% Là je sais pas mais on pourrait faire les fonctions des roues, l'algo pour le chemin, le multi threading, le code pour le son ...

\subsection{Développement de l'interface de contrôle}
% Ici aussi on pourrait détailler succintement avec la communication serveur/robot, la manette, la soundboard et les réglages de vitesse du robot + l'affichage au besoin des logs, de la caméra etc ...
\section{Mécanique}
\subsection{Choix de la taille des roues}
\subsection{Emplacement des roues et des capteurs}

% A section with detailing each member’s taks in the project + l'organisation interne
\chapter{Méthodologie}
\lipsum[5-6]

\chapter{Résultats}
\lipsum[7-8]
\section{Difficultés rencontrées}
\subsection{Problèmes de son}
\subsection{Échec des capteurs QTR}
\subsection{Tentative d'algorithme suiveur de ligne PID}
\subsection{Retours d'expériences}
% Partie dans laquelle on revient sur ces difficultés, on essaie de les raisonner
% (leur trouver une raison) et faire comme si le projet avait été une source incroyable d'apprentissage pour nous
\section{Les capacités du robot}

\chapter{Conclusion}
\lipsum[9-10]

% Bibliographie
\begin{thebibliography}{9}
\bibitem{latexcompanion} 
Michel Goossens, Frank Mittelbach, and Alexander Samarin. 
\textit{The \LaTeX\ Companion}. 
Addison-Wesley, Reading, Massachusetts, 1993.

\bibitem{einstein} 
Albert Einstein. 
\textit{Zur Elektrodynamik bewegter K{\"o}rper}. (German) 
[\textit{On the electrodynamics of moving bodies}]. 
Annalen der Physik, 322(10):891–921, 1905.

\bibitem{knuthwebsite} 
Knuth: Computers and Typesetting,
\\\texttt{http://www-cs-faculty.stanford.edu/\~{}uno/abcde.html}
\end{thebibliography}

\end{document}
