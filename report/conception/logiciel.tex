% A detailed section of your code: approach, structure, etc.
\section{Logiciel}

\subsection{Architecture du programme}

Le développement des fonctions métiers du robot représente une étape essentielle au sein de notre projet.
En effet, nous avons dû développer des fonctions permettant de contrôler les différents composants de notre robot pour répondre à notre cahier des charges.

Pour garder le développement structuré, nous avons décidé de découper notre programme en plusieurs parties, généralement concrétisées par des classes.

Voici les différentes fonctions que nous avons développées :
\begin{itemize}
    \item Une couche d'abstraction matérielle.
    \item Journalisation des événements.
    \item Fonctions permettant de contrôler les moteurs.
    \item Fonctions permettant de contrôler l'écran LCD.
    \item Fonctions permettant de contrôler le capteur de distance lidar.
    \item Fonctions permettant de contrôler le haut-parleur.
    \item Fonctions permettant de récupérer la position de la ligne.
    \item Fonctions permettant de contrôler le robot à distance à l'aide d'une interface de contrôle.
    \item Fonctions permettant au robot de se déplacement automatiquement.
\end{itemize}

\subsubsection{Couche d'abstraction matérielle}

La couche d'abstraction matérielle nous a permis de tester notre programme en intégration continue / machine locale ou sur le Raspberry Pi. 
Elle est basée sur des macros du préprocesseur C++.
Dans le cas d'une compilation sur le Raspberry Pi, les fonctions de la couche d'abstraction matérielle sont remplacées par des fonctions permettant de contrôler les composants de notre robot.
Tandis que dans le cas d'une compilation en machine locale, les fonctions de la couche d'abstraction matérielle sont remplacées des messages de logs.
Ainsi, à la précompilation, le code est adapté à la plateforme cible (natif ou Raspberry Pi), et peut être testé sans avoir besoin d'un Raspberry Pi.

Les fichiers concernés sont \texttt{src/pin.h} et \texttt{src/pin.cpp}.

\subsubsection{Journalisation des événements}

Afin de pouvoir déboguer notre programme aisément, nous avons développé un système de journalisation des événements.
Ainsi, nous pouvons afficher des messages de logs sur la sortie standard de notre programme avec des couleurs différentes en fonction du niveau de criticité du message (erreur, avertissement, information, débogage et trace).
De plus, l'affichage peut être réglé à la compilation en modifiant une macro (\texttt{LOG\_LEVEL}).

\subsubsection{Contrôle des moteurs}

Une classe Moteur a été créée afin de contrôler les moteurs de notre robot. Cette classe permet de contrôler la vitesse et la direction de nos moteurs.
Nous envoyons simplement des commandes au pont en H qui lui se charge de contrôler les moteurs.

\subsubsection*{Contrôle de l'écran LCD}
Une classe LCD a été créée afin de contrôler l'écran LCD de notre robot. Cette classe permet d'afficher des informations sur l'écran LCD de notre robot.
Nous envoyons des simples commandes au PCF8574T qui lui se charge de contrôler l'écran LCD.

\subsubsection*{Capteur de distance lidar}

Une classe Lidar a été créée afin de contrôler le capteur de distance lidar de notre robot.
Cette classe permet de récupérer la distance mesurée par le capteur de distance lidar en lisant et écrivant les registres du capteur avec le protocole I2C.

\subsubsection*{Contrôle du haut-parleur}

Pour envoyer des sons sur notre haut-parleur nous lançons simplement des commandes systèmes qui permettent de jouer des sons sur notre Raspberry Pi qui sont ensuite envoyés sur notre haut-parleur.
\\
L'interface web est capable d'envoyer des commandes au robot pour jouer des sons.

\subsubsection*{Récupération de la position de la ligne}
Pour récupérer la position de la ligne nous utilisons une caméra. La caméra est en mode noir et blanc et nous avons séparé l'image en 5 zones.
Nous faisons la moyenne des pixels de chaque zone et si cette moyenne est supérieure à un certain seuil alors nous considérons que la ligne noire est dans cette zone. La ligne noire peut se trouver dans plusieurs zones simultanément.



\subsubsection{Développement de l'interface de contrôle}
Nous avons décidé de nous tourner vers les technologies web pour développer l'interface de contrôle de notre robot. Celles-ci, par leur simplicité, nous ont permis d'avancer rapidement, et de nous concentrer plutôt sur le développement des fonctionnalités de notre robot. Les navigateurs offrent notamment un large support des manettes de jeux, et son intégration a donc été très simple.
Le robot fait tourner un serveur HTTP afin de recevoir les commandes du client. Internet étant accessible à travers le wifi de l'INSA, le robot peut donc être controlé à distance dans tout le périmètre de l'école. Disposant d'une caméra, il est possible de s'aider du flux vidéo pour contrôler le robot.
Néanmoins, le réseau de l'INSA nous faire subir quelques instabilités. Si la latence est très satisfaisante le plus souvent, il arrive au robot de perdre la connexion avec le client pendant quelques secondes. Ces problèmes de perte de contrôle du véhicule peuvent se révéler très problématiques. Nous désignons la DSI comme responsable d'éventuels accidents.
L'interface web nous a été très utile pour piloter le robot, mais aussi pour expérimenter avec différents paramètres, ainsi que pour visualiser l'état de certains capteurs. La plupart des paramètres et affichages des entrées ont été supprimées aujourd'hui.

\subsubsection*{Contrôle du robot à distance}
L'interface web envoie continuellement la position du joystick de la manette au robot. A partir de cette information, le robot calcule par une simple formule la vitesse de chaque moteur. Afin de réduire les problèmes de latence, l'interface web s'assure de ne pas envoyer les commandes plus rapidement que le robot ne les traite.

\subsubsection*{Contrôle automatique du robot}
Nous pouvons également changer le mode de fonctionnement de notre robot depuis l'interface web. Ce dernier peut ainsi entrer dans le mode "suiveur de ligne", se mettant ainsi à chercher et suivre automatiquement toute ligne noire présente en face de lui.