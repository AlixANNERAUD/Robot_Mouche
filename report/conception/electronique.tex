% A complete fritzing circuit with all the wiring of the different sensors, actuators and components for display
% + A detailed explanation of your wiring choices as well as any additional technical choices (power, use of resistors, diodes, etc.)
% + A detailed explanation of the chosen components role and operating principle
\section{Electronique}
\subsection{Choix des composants}
Le choix des composants a été une étape importante dans la réalisation de notre robot. En effet, nous avons dû choisir des composants qui correspondaient à nos besoins et qui étaient compatibles avec notre Raspberry Pi. 

Voici les différents composants que nous avons utilisés :
\begin{itemize}
    \item L293D, un pont en H, pour contrôler nos moteurs
    \item 1602A, un écran LCD, pour afficher des informations sur notre robot
    \item PCF8574T pour contrôler notre écran LCD à travers le bus I2C permettant d'économiser des pins sur notre Raspberry Pi
    \item SJ-GU-TF-Luna un capteur de distance lidar pour détecter les obstacles, ce capteur est plus précis que le capteur de distance à ultrasons.
    \item Une batterie externe pour alimenter notre Raspberry Pi
    \item Une batterie pour alimenter nos moteurs
    \item Un amplificateur audio pour pouvoir jouer des sons sur notre robot
    \item Un haut-parleur pour émettre des sons
    \item Caméra pour repérer la position de la ligne
\end{itemize}

\subsection{Cablâge des composants}

\todo

\subsection{Schéma EasyEDA}

\todo